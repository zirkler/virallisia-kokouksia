\documentclass[a4paper, twocolumn]{article}
\setlength{\oddsidemargin}{0cm}
\setlength{\evensidemargin}{0cm}
\setlength{\topmargin}{0cm}
\usepackage{indentfirst}
\usepackage[utf8]{inputenc}
\usepackage{amsmath}
\usepackage{amssymb}

\begin{document}

\section{Vektorgeometrie}
	\paragraph{Winkel zwischen zwei Vektoren:} Vektoren $u$ und $v$ schließen den Winkel $\varphi$ ein. Dann gilt:
	$cos \varphi = \frac{\langle u, v \rangle}{|u| \cdot |v|}$	
	
	\paragraph{Schnittwinkel zwischen Gerade und Ebene}
	$= 90^\circ - cos \varphi = \frac{\langle n, r \rangle}{|n| \cdot |r|}$	

\subsection{Geraden}
    Gerade in Parameterform:
    \begin{displaymath}
       g\colon x(\lambda) = a + \lambda v
    \end{displaymath}


\subsection{Ebenen}
    Ebene in Parameterform (Punkt-Richtungs-Form):
    \begin{displaymath}
       E:	x(\lambda, \mu) = a + \lambda r_1 + \mu r_2
    \end{displaymath}
    
    Ebene in Normalform:
    \begin{displaymath}
       E:	\langle x - a, n \rangle = 0
    \end{displaymath}
    
    Ebene in Koordinatenform:
    \begin{displaymath}
       E: n_1 \cdot x + n_2 \cdot y + n_3 \cdot z = a
    \end{displaymath}

   \subsubsection{Umformen}
   	\paragraph{Normalform nach Koordinatenform:}
        \begin{flalign*}
            \langle \begin{pmatrix} a_1 \\ a_2 \\ a_3 \end{pmatrix} , \begin{pmatrix} x_1 \\ x_2 \\ x_3 \end{pmatrix} - \begin{pmatrix} n_1 \\ n_2 \\ n_3 \end{pmatrix} \rangle = 0 \nonumber \\ 
            \Rightarrow  \langle \begin{pmatrix} a_1 \\ a_2 \\ a_3 \end{pmatrix} , \begin{pmatrix} x_1 \\ x_2 \\ x_3 \end{pmatrix} \rangle - \langle \begin{pmatrix} a_1 \\ a_2 \\ a_3 \end{pmatrix} , \begin{pmatrix} n_1 \\ n_2 \\ n_3 \end{pmatrix} \rangle = 0 \nonumber \\
            \Rightarrow a_1 x_1 + a_2 x_2 + a_3 x_3 - (a_1 n_1 + a_2 n_2 + a_3 n_3 ) = 0 \nonumber
        \end{flalign*}
        
        \paragraph{Parameterform nach Normalform}
        \begin{align*}
        	geg\colon E\colon (\lambda, \mu) = a + \lambda \cdot r_1 + \mu \cdot r_2 \\
		n = r_1 \times r_2 
		\Rightarrow \langle x - a, n \rangle = 0 
        \end{align*}
        (a ist Aufpunkt der Ebene)
        
        \paragraph{Koordinatenform nach Normalform: }
        Normalenvektor ablesen:
        \begin{flalign*}
        	geg\colon -x_1 + 2x_2 - x_3 = 0 \\
		\Rightarrow n =  \begin{pmatrix} -1 \\ 2 \\ -1 \end{pmatrix}
        \end{flalign*}
        Nun muss man einen Punkt finden, welcher in der Ebene liegt, also die Koordinatengleich erfüllt.
	Dieser Punkt ist $a$ in $\langle x - a, n \rangle = 0$. Fertig.
	
	\paragraph{Koordinatenform nach Parameterform: }
	Man muss drei Punkte $a, r_1, r_2$ in der Ebene finden, diese werden dann in $x(\lambda, \mu) = a + \lambda r_1 + \mu r_2$ eingesetzt.
   
   \subsubsection{Lagebeziehungen} 
   	\paragraph{Ebene zu Gerade}
	Eine Ebene ist zu einer Gerade parallel, wenn der Normalenvektor der Ebene orthogonal zum Richtungsvektor der Geraden ist (Skalarprodukt = 0).
	
   \subsubsection{Schnitte} 
       \paragraph{Schnittpunkt zweier Geraden}
       dazu müssen die beiden Richtungsvektoren der Geraden linear unabhängig sein. Dann wird einfach $g = h$ gesetzt.
       
       \paragraph{Schnittpunkt Gerade mit Ebene}
	$x(t_o) = a + t_0 \cdot r$, so dass $\langle n, x(t_0) \rangle = \delta_E$ \\
	$ \Rightarrow t_0 = \frac{\delta_E - \langle n, a \rangle}{\langle n , r \rangle} $
	
       
   
   \subsubsection{Abstände}
      \paragraph{Punkt zu Punkt:}
      Abstand von P zu Q $= |P-Q|$
      
      \paragraph{Punkt zu Gerade:}
      gegeben sei die Gerade $g\colon x = a + \lambda r$ mit $||r|| = 1$ und Punkt $P$. 
      Man wählt bel. Punkt B auf $g$. Setze $\omega := p-b$. 
      Es Sei $\varphi := \sphericalangle(\omega, r)$. Dann gilt:\\
      Abstand $\delta = ||\omega|| \cdot sin \varphi = ||\omega|| \cdot ||r|| \cdot sin \varphi = ||\omega \times r||$
      
      \paragraph{Abstand zwischen zwei parallelen Geraden:}
      Wähle Punkt $P$ auf $g_1$, dann wie oben $\delta$ berechnen.

      \paragraph{Abstand zwischen zwei windschiefen Geraden:}
      Die Geraden $g: x = p +  tu$ und $h: x = q + sv$ seien windschief. $n = u \times v$ (n steht senkrecht auf u und v). Dann gilt: $\delta = \frac{|(q-p) \cdot n|}{|n|}$.

      \paragraph{Punkt zu Ebene}      
      Hilfsgerade $h: x = p + t \cdot n$ durch P, die senkrecht auf der Ebene $E$ steht. Wo sich $h$ und $E$ schneiden ist der Fußpunkt F. Abstand $\delta = |PF|$.

      \paragraph{Gerade zu Ebene}
      Beliebigen Punkt auf der Geraden (alle Punkte haben den gleichen Abtand) in $|HNF|$ mit ($|n| = 1$) einsetzen.
      
      \paragraph{Ebene zu Ebene}
   
 
\end{document}















    